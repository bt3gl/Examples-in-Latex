\section{Introduction}

\begin{frame}
  \center{\fcolorbox{green}{white}{\bf Introduction}}
\begin{center} {\scriptsize
{\bf Neutron Stars: What we know and what we do not know.}}
\end{center}
\end{frame}



%%%%%%%%%%%%%%%%%%%%%%%%%%%%%%%%%%%%%%%%%%%%%%%%%%%%%%%%%%%%%%%%%%%%%%%%%%%%%%%%%%%
\subsection*{Physics of NS}

\begin{frame}
\frametitle{Physics of NS}  
 % \begin{columns}[c]
  %  \begin{column}{0.6\textwidth}  
	{\scriptsize 
	  \begin{itemize}
	   \item More than 2000 NSs have been discovered in the Galaxy, from pulsating sources in radio to bright persistent sources in X-Rays.
	   \item Stellar objects with most extreme densities and strongest magnetic fields.
	   \item Strongest gravitational fields among all objects with a surface.
	  \end{itemize}
	}       
  %  \end{column}
   % \begin{column}{0.5\textwidth}    
      % \fcolorbox{white}{white}{\includegraphics[scale=0.27]{figs/qgp/as_freed.png}}
	%\center{\tiny{(Bethke (2009), [hep-ph] 0908.1135.)}}
    %\end{column}
  %\end{columns}    
\end{frame}


\begin{frame}
\frametitle{Physics of NS}  
 % \begin{columns}[c]
  %  \begin{column}{0.6\textwidth}  
	{\scriptsize 
	  \begin{itemize}
	   \item Studying observed surface emission as a tool to understand NS requires detailed models of their atmospheres, which shape the spectrum and the pattern of radiation from the stellar crust and core:
	   \begin{itemize}
	    \item  many magnetic field strengths,
	    \item compositions,
	    \item temperatures,
	    \item ionization states.
	   \end{itemize}
	   \item Observations of NS in quiescence and during the thermonuclear bursts led to first constraining measurements of radii, $9 km < R <12 km$ (eos >15km is disfavored by observations).
	  \end{itemize}
	}       
  %  \end{column}
   % \begin{column}{0.5\textwidth}    
      % \fcolorbox{white}{white}{\includegraphics[scale=0.27]{figs/qgp/as_freed.png}}
	%\center{\tiny{(Bethke (2009), [hep-ph] 0908.1135.)}}
    %\end{column}
  %\end{columns}    
\end{frame}

